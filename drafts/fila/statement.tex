\documentclass{article}

\begin{document}
\title{Fila de supermercado}
\maketitle

Hay una fila en el supermercado.
A priori se sabe cuanto se va a demorar cada persona en ser atendida en la caja.
De pronto se abre la caja del lado y es posible que algunas personas se muevan a esa fila.
Hay que hacerlo de forma justa.
La forma más justa es la que preserva el orden en que las personas hubieran sido atendidas si solo se hubiera mantenido la primera caja.

Posibles modificaciones
\begin{itemize}
  \item  Generalizarlo a $n$ cajas nuevas.
  \item Hacer el problema m\'as ``online'' donde solo se sabe cuanto se demoran en atender a una persona en el momento en que la están atendiendo y agregando que las personas se pueden cambiar un número limitado de veces de fila.
\end{itemize}


\end{document}