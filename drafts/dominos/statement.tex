\documentclass{article}
\usepackage[utf8]{inputenc}
\usepackage{fullpage}

\title{Dominós\footnote{Problema inspirado en \emph{Matemáticas Recreativas} de Y. I. Perelman}}

\begin{document}
\maketitle

Un juego de dominós consiste en un conjunto de piezas rectangulares de $2\times1$, donde cada mitad tiene un número entre 0 y $n-1$, donde habitualmente $n$ es 7. %TODO será mejor poner una figura?
Llamaremos juego de $n$-dominós al juego de dominós con el valor correspondiente para $n$, por ejemplo un juego de $7$-dominós es aquél cuyas piezas tienen valores entre 0 y 6.
Además, hay una pieza para cada combinación, y toda combinación aparece exactamente una vez (1--3 y 3--1 son la misma pieza).
Por ejemplo, un juego de \textbf{7-dominós} consiste en \textbf{28 piezas}.

Un juego de dominós es similar a una baraja de cartas, en el sentido de que permiten jugar una gran variedad de juegos.
Uno de estos es el del cuadrado.
El juego consiste en formar, usando un juego completo de dominós, un cuadrado en cuyas esquinas haya 2 fichas verticales y 2 horizontales, y de forma tal que para cada lado, la suma de los valores en él sea $k$.
Llamaremos a este cuadrado el cuadrado $(n,k)$.
Por supuesto, no es posible completar este juego para todos los $n$ ni, dado $n$, para todos los $k$.

Tu objetivo es, entonces, determinar una configuración de piezas para completar el cuadrado $(n,k)$.
Para esto, tendrás que implementar algunas funciones que simplificarán enormemente la resolución del problema.

\section*{Tarea}

Debes implementar la función \verb+cuadrado+, que determina para un $n$ si es posible construir un cuadrado con un juego de $n-$dominós; no se pide que los lados sumen lo mismo, sólo se pide saber si se puede armar un cuadrado usando \emph{TODO} el juego de $n$-dominós.

Además debes implementar la función \verb+validar+ que determina si un cuadrado cumple con la restricción de que todos los lados sumen lo mismo.

Además, debes implementar la función \verb+construir+ que para $n$ y $k$ construye, si es posible, un cuadrado $(n,k)$, y retorna \verb+true+ si es posible o \verb+false+ si no.

\begin{itemize}
 \item \verb+cuadrado(n)+
 \begin{itemize}
  \item \verb+n+: el juego de $n$-dominós.
  \item \emph{return}: un booleano con valor \verb+true+ si se puede formar un cuadrado, y \verb+false+ si no.
 \end{itemize}
 \item \verb+validar(n, d)+
 \begin{itemize}
  \item \verb+n+: el juego de $n$-dominós.
  \item \verb+d+: arreglo de $2f$ enteros, donde se guarda el orden de las fichas que forma el cuadrado.
  \item \emph{return}: un booleano con valor \verb+true+ si se puede formar un cuadrado $(n,k)$, y \verb+false+ si no.
 \end{itemize}

 \item \verb+construir(n, k, d)+
 \begin{itemize}
  \item \verb+n, k+: valores para $n,k$, que indican el cuadrado $(n,k)$ que se quiere construir.
  \item \verb+d+: arreglo de $2f$ enteros, donde usted debe guardar los valores de las fichas de un cuadrado $(n,k)$, partiendo desde alguna esquina. Observe que $f$ es el número de piezas de dominó. %TODO agregar figura?
  \item \emph{return}: un booleano con valor \verb+true+ si se puede formar un cuadrado $(n,k)$, y \verb+false+ si no.
 \end{itemize}
\end{itemize}

\section*{Subtareas}
\begin{itemize}
  \item (a puntos) Implementar la función cuadrado.
  \item (b puntos) Implementar la función validar.
  \item (c puntos) Implementar la función construir, con $n=7$.
  \item (d puntos) Implementar la función construir, para $n \le 16$.
  \item (e puntos) Implementar la función construir, para $n \le 1000$.
\end{itemize}

%TODO detalles de implementación: todo es int

%TODO ejemplo (particularmente figura) con $n=7, k=42$.

\end{document}
