\documentclass{oci}
\usepackage[utf8]{inputenc}
\usepackage{tikz}
\usepackage{subcaption}
\usepackage{lipsum}
\usepackage[spanish]{babel}

\title{Dominó}

\begin{document}
\maketitle

% Un juego completo de fichas de dominó consiste en un conjunto de piezas rectangulares de $2\times1$, divididas en dos cuadrados, donde cada mitad tiene un número entre 0 y $n-1$, donde habitualmente $n$ es 7.
Un juego de fichas de dominó consiste en un conjunto de piezas rectangulares de $2\times1$, donde cada mitad contiene un número entre 0 y $n-1$. 
% En el set del dominó habitual $n$ es igual 7.
Llamaremos un $n$-dominó al juego de fichas donde el valor correspondiente es $n$.
Por ejemplo, el juego de dominó normal es un $7$-dominó y sus piezas tienen valores entre 0 y 6.
Además, en un $n$-dominó cada combinación de pares de números está presente exactamente una vez (1--3 y 3--1 son la misma pieza).
Por ejemplo, un \textbf{7-dominó} consiste en \textbf{28 piezas}.

Un $n$-dominó es tan versátil como una baraja de cartas, en cuanto a que ambos permiten jugar una gran variedad de juegos.
Uno de estos es el del cuadrado\footnote{Inspirado en \emph{Matemáticas Recreativas} de Y. I. Perelman}.
En el juego del cuadrado se te entrega un $n$-dominó y un valor $k$.
El objetivo del juego es formar un cuadrado usando todas las piezas del $n$-dominó, de forma que para cada lado la suma de todos los valores en él sea $k$.
% Este juego\footnote{Inspirado en \emph{Matemáticas Recreativas} de Y. I. Perelman} consiste en formar un cuadrado usando todas las piezas de un $n$-dominó,  donde el número de piezas verticales y horizontales es el mismo (fig. \ref{f:cuadrados}), tal que para cada lado, la suma de los valores en él sea $k$.
% En este juego está permitido poner juntos dos cuadrados de diferente valor (a diferencia del clásico juego de dominó).
A un cuadrado formado con un $n$-dominó de forma que los lados sumen $k$ lo llamaremos un cuadrado $(n,k)$.

Para formar un cuadrado válido la cantidad de pieza verticales y horizontales debe ser la misma.
A continuación se muestra una imagen de un cuadrado inválido y uno válido.
\begin{figure}[h!]
 \begin{subfigure}[b]{0.45\textwidth}
  \centering
  \begin{tikzpicture}[scale=.8]
\def\one{+(0.00,0.00) circle [radius=0.08]}
\def\two{+(-.25, .25) circle [radius=0.08]
	 +( .25,-.25) circle [radius=0.08]}
\def\thr{+(-.25, .25) circle [radius=0.08]
	 +(0.00,0.00) circle [radius=0.08]
	 +( .25,-.25) circle [radius=0.08]}
\def\fou{+(-.25,-.25) circle [radius=0.08]
	 +(-.25, .25) circle [radius=0.08]
	 +( .25,-.25) circle [radius=0.08]
	 +( .25, .25) circle [radius=0.08]}
\def\fiv{+(-.25,-.25) circle [radius=0.08]
	 +(-.25, .25) circle [radius=0.08]
	 +(0.00,0.00) circle [radius=0.08]
	 +( .25,-.25) circle [radius=0.08]
	 +( .25, .25) circle [radius=0.08]}
\def\six{+(-.25,-.25) circle [radius=0.08]
	 +(-.25, .25) circle [radius=0.08]
	 +(0.00,-.25) circle [radius=0.08]
	 +(0.00, .25) circle [radius=0.08]
	 +( .25,-.25) circle [radius=0.08]
	 +( .25, .25) circle [radius=0.08]}
\def\domino{
  [black,rounded corners=3.2pt]+(-1,-.5) rectangle +(1,.5)
  [gray,thin]+(0,.35) -- +(0,-.35)
}
\draw ( 0, 1.5) \domino;
\begin{scope}[rotate=90]
  \draw (-1, 1.5) \domino;
  \draw ( 1, 1.5) \domino;
\end{scope}
\begin{scope}[rotate=180]
  \draw (0, 1.5) \domino;
\end{scope}
\begin{scope}[rotate=270]
  \draw (-1, 1.5) \domino;
  \draw ( 1, 1.5) \domino;
\end{scope}
\end{tikzpicture}

  \caption{Cuadrado inválido}
  \label{f:invalido}
 \end{subfigure}
 \begin{subfigure}[b]{0.45\textwidth}
  \centering
  \begin{tikzpicture}[scale=.8]
\def\one{+(0.00,0.00) circle [radius=0.08]}
\def\two{+(-.25, .25) circle [radius=0.08]
	 +( .25,-.25) circle [radius=0.08]}
\def\thr{+(-.25, .25) circle [radius=0.08]
	 +(0.00,0.00) circle [radius=0.08]
	 +( .25,-.25) circle [radius=0.08]}
\def\fou{+(-.25,-.25) circle [radius=0.08]
	 +(-.25, .25) circle [radius=0.08]
	 +( .25,-.25) circle [radius=0.08]
	 +( .25, .25) circle [radius=0.08]}
\def\fiv{+(-.25,-.25) circle [radius=0.08]
	 +(-.25, .25) circle [radius=0.08]
	 +(0.00,0.00) circle [radius=0.08]
	 +( .25,-.25) circle [radius=0.08]
	 +( .25, .25) circle [radius=0.08]}
\def\six{+(-.25,-.25) circle [radius=0.08]
	 +(-.25, .25) circle [radius=0.08]
	 +(0.00,-.25) circle [radius=0.08]
	 +(0.00, .25) circle [radius=0.08]
	 +( .25,-.25) circle [radius=0.08]
	 +( .25, .25) circle [radius=0.08]}
\def\domino{
  [black,rounded corners=3.2pt]+(-1,-.5) rectangle +(1,.5)
  [gray,thin]+(0,.35) -- +(0,-.35)
}
\draw (-1.5, 2.0) \domino;
\draw ( 0.5, 2.0) \domino;
\begin{scope}[rotate=90]
  \draw (-1.5, 2.0) \domino;
  \draw ( 0.5, 2.0) \domino;
\end{scope}
\begin{scope}[rotate=180]
  \draw (-1.5, 2.0) \domino;
  \draw ( 0.5, 2.0) \domino;
\end{scope}
\begin{scope}[rotate=270]
  \draw (-1.5, 2.0) \domino;
  \draw ( 0.5, 2.0) \domino;
\end{scope}
\end{tikzpicture}

  \caption{Cuadrado de lado 5}
  \label{f:valido}
 \end{subfigure}
 \caption{Un cuadrado válido y uno inválido.}
 \label{f:cuadrados}
\end{figure}
Notar que no para cualquier valor de $n$ es posible formar un cuadrado con un $n$-dominó.
Por ejemplo, el $2$-dominó está formado por 3 piezas (0--0, 0--1 y 1--1) y no es posible armar un cuadrado con ellas.
Por otro lado, para un $n$-dominó donde sí es posible formar un cuadrado no necesariamente es posible armar un cuadrado $(n,k)$ para cualquier $k$.
Por ejemplo, para el $7$-dominó la suma de los lados siempre será menor que 90, y por lo tanto no es posible formar un cuadrado $(7,90)$.

\begin{figure}[ht]
 \centering
 \begin{tikzpicture}[scale=.4]
\def\one{+(0.00,0.00) circle [radius=0.08]}
\def\two{+(-.25, .25) circle [radius=0.08]
	 +( .25,-.25) circle [radius=0.08]}
\def\thr{+(-.25, .25) circle [radius=0.08]
	 +(0.00,0.00) circle [radius=0.08]
	 +( .25,-.25) circle [radius=0.08]}
\def\fou{+(-.25,-.25) circle [radius=0.08]
	 +(-.25, .25) circle [radius=0.08]
	 +( .25,-.25) circle [radius=0.08]
	 +( .25, .25) circle [radius=0.08]}
\def\fiv{+(-.25,-.25) circle [radius=0.08]
	 +(-.25, .25) circle [radius=0.08]
	 +(0.00,0.00) circle [radius=0.08]
	 +( .25,-.25) circle [radius=0.08]
	 +( .25, .25) circle [radius=0.08]}
\def\six{+(-.25,-.25) circle [radius=0.08]
	 +(-.25, .25) circle [radius=0.08]
	 +(0.00,-.25) circle [radius=0.08]
	 +(0.00, .25) circle [radius=0.08]
	 +( .25,-.25) circle [radius=0.08]
	 +( .25, .25) circle [radius=0.08]}
\def\domino{
  [rounded corners=1.6pt]+(-1,-.5) rectangle +(1,.5)
  [gray,thin]+(0,.35) -- +(0,-.35)
}
\draw (-6.5, 7.0) \domino;
\fill (-6.5, 7.0) ++(-0.5,0) \six;
\fill (-6.5, 7.0) ++( 0.5,0) \two;
\draw (-4.5, 7.0) \domino;
%\fill (-4.5, 7.0) ++(-0.5,0) \six;
\fill (-4.5, 7.0) ++( 0.5,0) \fou;
\draw (-2.5, 7.0) \domino;
\fill (-2.5, 7.0) ++(-0.5,0) \fiv;
\fill (-2.5, 7.0) ++( 0.5,0) \one;
\draw (-0.5, 7.0) \domino;
\fill (-0.5, 7.0) ++(-0.5,0) \fou;
\fill (-0.5, 7.0) ++( 0.5,0) \six;
\draw ( 1.5, 7.0) \domino;
\fill ( 1.5, 7.0) ++(-0.5,0) \thr;
\fill ( 1.5, 7.0) ++( 0.5,0) \one;
\draw ( 3.5, 7.0) \domino;
\fill ( 3.5, 7.0) ++(-0.5,0) \one;
%\fill ( 3.5, 7.0) ++( 0.5,0) \six;
\draw ( 5.5, 7.0) \domino;
\fill ( 5.5, 7.0) ++(-0.5,0) \fou;
\fill ( 5.5, 7.0) ++( 0.5,0) \fiv;
\begin{scope}[rotate=90]
  \draw (-6.5, 7.0) \domino;
  \fill (-6.5, 7.0) ++(-0.5,0) \two;
  \fill (-6.5, 7.0) ++( 0.5,0) \thr;
  \draw (-4.5, 7.0) \domino;
  \fill (-4.5, 7.0) ++(-0.5,0) \thr;
%  \fill (-4.5, 7.0) ++( 0.5,0) \six;
  \draw (-2.5, 7.0) \domino;
%  \fill (-2.5, 7.0) ++(-0.5,0) \six;
  \fill (-2.5, 7.0) ++( 0.5,0) \six;
  \draw (-0.5, 7.0) \domino;
%  \fill (-0.5, 7.0) ++(-0.5,0) \six;
  \fill (-0.5, 7.0) ++( 0.5,0) \two;
  \draw ( 1.5, 7.0) \domino;
  \fill ( 1.5, 7.0) ++(-0.5,0) \six;
  \fill ( 1.5, 7.0) ++( 0.5,0) \fiv;
  \draw ( 3.5, 7.0) \domino;
  \fill ( 3.5, 7.0) ++(-0.5,0) \two;
  \fill ( 3.5, 7.0) ++( 0.5,0) \two;
  \draw ( 5.5, 7.0) \domino;
  \fill ( 5.5, 7.0) ++(-0.5,0) \fiv;
  \fill ( 5.5, 7.0) ++( 0.5,0) \thr;
\end{scope}
\begin{scope}[rotate=180]
  \draw (-6.5, 7.0) \domino;
  \fill (-6.5, 7.0) ++(-0.5,0) \one;
  \fill (-6.5, 7.0) ++( 0.5,0) \one;
  \draw (-4.5, 7.0) \domino;
  \fill (-4.5, 7.0) ++(-0.5,0) \thr;
  \fill (-4.5, 7.0) ++( 0.5,0) \six;
  \draw (-2.5, 7.0) \domino;
  \fill (-2.5, 7.0) ++(-0.5,0) \two;
  \fill (-2.5, 7.0) ++( 0.5,0) \fou;
  \draw (-0.5, 7.0) \domino;
  \fill (-0.5, 7.0) ++(-0.5,0) \one;
  \fill (-0.5, 7.0) ++( 0.5,0) \two;
  \draw ( 1.5, 7.0) \domino;
  \fill ( 1.5, 7.0) ++(-0.5,0) \fou;
  \fill ( 1.5, 7.0) ++( 0.5,0) \fou;
  \draw ( 3.5, 7.0) \domino;
  \fill ( 3.5, 7.0) ++(-0.5,0) \one;
  \fill ( 3.5, 7.0) ++( 0.5,0) \fou;
  \draw ( 5.5, 7.0) \domino;
  \fill ( 5.5, 7.0) ++(-0.5,0) \fiv;
  \fill ( 5.5, 7.0) ++( 0.5,0) \fiv;
\end{scope}
\begin{scope}[rotate=270]
  \draw (-6.5, 7.0) \domino;
  \fill (-6.5, 7.0) ++(-0.5,0) \thr;
  \fill (-6.5, 7.0) ++( 0.5,0) \fou;
  \draw (-4.5, 7.0) \domino;
%  \fill (-4.5, 7.0) ++(-0.5,0) \fou;
%  \fill (-4.5, 7.0) ++( 0.5,0) \two;
  \draw (-2.5, 7.0) \domino;
  \fill (-2.5, 7.0) ++(-0.5,0) \fiv;
%  \fill (-2.5, 7.0) ++( 0.5,0) \six;
  \draw (-0.5, 7.0) \domino;
  \fill (-0.5, 7.0) ++(-0.5,0) \fiv;
  \fill (-0.5, 7.0) ++( 0.5,0) \two;
  \draw ( 1.5, 7.0) \domino;
  \fill ( 1.5, 7.0) ++(-0.5,0) \six;
  \fill ( 1.5, 7.0) ++( 0.5,0) \one;
  \draw ( 3.5, 7.0) \domino;
  \fill ( 3.5, 7.0) ++(-0.5,0) \six;
  \fill ( 3.5, 7.0) ++( 0.5,0) \six;
  \draw ( 5.5, 7.0) \domino;
  \fill ( 5.5, 7.0) ++(-0.5,0) \thr;
  \fill ( 5.5, 7.0) ++( 0.5,0) \thr;
\end{scope}
\end{tikzpicture}

 \caption{Un cuadrado $(7,45)$. Notar que cada lado suma 45.}
 \label{f:cuadrado7}
\end{figure}

Tu objetivo es, entonces, determinar una configuración de piezas para completar el cuadrado $(n,k)$.
Para esto, tendrás que implementar algunas funciones que simplificarán enormemente la resolución del problema.



\subsection*{Subtarea 1 (25 pts)}
La primera subtarea consiste en implementar la función \verb+cuadrado+ que determina para un $n$ si es posible formar un cuadrado con un $n$-dominó.
En esta subtarea no importa la suma de los valores de cada lado, sólo se pide determinar si es posible formar un cuadrado válido usando todas las piezas del $n$-dominó.

\begin{itemize}
 \item \verb+bool cuadrado(int n)+
 \begin{itemize}
  \item \verb+n+: indica el tamaño del $n$-dominó.
  \item \emph{return}: un booleano con valor \verb+true+ si se puede formar un cuadrado, y \verb+false+ si no.
 \end{itemize}
\end{itemize}

Para esta subtarea se probará la función \verb+cuadrado+ con varios casos donde $N\leq 1000$ (25pts).

\subsection*{Subtarea 2 (20 pts)}
La subtarea 2 consiste en implementar la función \verb+validar+ que dada la especificación de un cuadrado formado por fichas de un $n$-dominó debe verificar que este cumple con la restricción de que todos los lados sumen lo mismo.
Esto también implica verificar que las fichas usadas correspondan a las del $n$-dominó, es decir, cada ficha es usada exactamente una vez.
Puedes asumir que para el $n$-dominó entregado es posible formar un cuadrado válido, es decir, \verb+cuadrado(n)+ retorna \verb+true+.

\begin{itemize}
 \item \verb+bool validar(int n, int f, int fichas[])+
 \begin{itemize}
  \item \verb+n+: indica el tamaño del $n$-dominó.
  \item \verb+f+: indica la cantidad de fichas que tiene el $n$-dominó.
  \item \verb+fichas+: arreglo de enteros de tamaño $2\times\texttt{f}$ donde se especifica el cuadrado. Más adelante se detalla el formato en que el cuadrado debe ser especificado.
  \item \emph{return}: un booleano con valor \verb+true+ si los cuatro lados suman lo mismo, y \verb+false+ si no.
 \end{itemize}
\end{itemize}

Para esta subtarea se probará la función \verb+validar+ con varios casos donde $N\leq 1000$ (20pts).

\subsection*{Subtarea 3 y 4}
Para las subtareas 3 y 4, debes implementar la función \verb+construir+ que para un $n$ y un $k$ dados construye un cuadrado $(n,k)$ en caso de ser posible.
Puedes asumir que para el $n$-dominó entregado es posible formar un cuadrado válido, es decir, \verb+cuadrado(n)+ retorna \verb+true+.
No obstante, puede que ser posible que para el valor de $k$ no sea posible formar un cuadrado $(n,k)$.
Si es posible formar un cuadrado $(n,k)$ la función debe retornar \verb+true+, en caso contrario debe retornar \verb+false+.
\begin{itemize}
 \item \verb+bool construir(int n, int k, int f, int fichas[])+
 \begin{itemize}
  \item \verb+n, k+: valores que indican el cuadrado $(n,k)$ que se quiere construir.
  \item \verb+f+: indica la cantidad de fichas que tiene el $n$-dominó.
  \item \verb+d+: arreglo de enteros de tamaño $2\times\texttt{f}$ donde debes guardar la especificación del cuadrado $(n,k)$. Más adelante se detalla la forma de especificar el cuadrado.
    Tu función debe llenar este arreglo. Su contenido será revisado sólo en el caso en que la función retorne \verb+true+.
  \item \emph{return}: un booleano con valor \verb+true+ si se puede formar un cuadrado $(n,k)$, y \verb+false+ si no.
 \end{itemize}
\end{itemize}

Para la subtarea 3 se probará la función \verb+validar+ con varios casos donde $N\leq 7$ (20pts).
Para la subtarea 4 se probará la función \verb+validar+ con varios casos donde $N\leq 1000$ (25pts).

\subsection*{Detalles de implementación}
Debes enviar exactamente un archivo, llamado \verb+domino.cpp+.
Este archivo deberá implementar todas las funciones descritas.
Si solo has implementado una de las funciones la otra debe también estar presente y puedes dejarla vacía.
El archivo \verb+domino.cpp+ debe también incluir el header \verb+domino.h+.

\subsection*{Grader de Ejemplo}
Se provee un \emph{grader} junto con algunos archivos de prueba para que puedas testear tu solución.
El grader lee de la entrada estándar en el siguiente formato.

La primera línea contiene un entero $T$ correspondiente a la subtarea que se quiere resolver.
El número $T$ puede ser 1, 2 o 3.
\begin{itemize}
 \item Si $T$ es 1, solo sigue una línea, que contiene un entero $N$ correspondiente al tamaño del $N$-dominó.
 \item Si $T$ es 2, siguen dos líneas.
La primera contiene dos enteros $N$ y $F$, correspondientes al tamaño y el número de fichas del $N$-dominó respectivamente.
La segunda línea contiene $2\times F$ enteros entre 0 y $N$, los valores del arreglo \emph{d} en el orden descrito abajo.
 \item Si $T$ es 3, sigue una sola línea, que contiene tres enteros $N$, $K$ y $F$, donde $N$ y $K$ indican el tipo de cuadrado que se pide construir, y $F$ es el número de fichas del $N$-dominó.
\end{itemize}

\subsection*{Convención sobre el formato del arreglo \emph{d}}
La función \verb+validar+ recibe como parámetro un arreglo de enteros \verb+fichas+ donde se especifica un cuadrado formado con las fichas de un $n$-dominó.
De la misma forma \verb+construir+ debe llenar el arreglo \verb+fichas+ para especificar el cuadrado que construye.
A continuación se detalla la forma en que un cuadrado es especificado en el arreglo \verb+fichas+.

Si $f$ es la cantidad de fichas de un $n$-dominó el arreglo \verb+fichas+ es de tamaño $2\times f$.
Por ejemplo, como un $7$-dominó tiene 28 fichas puede ser guardado en un arreglo de tamaño 56.
El arreglo es tal que dos posiciones seguidas corresponden a una ficha.
La posición 0 y la 1 corresponden a la primera ficha, la 2 y la 3 a la segunda ficha y así hasta la última ficha que corresponde a las posiciones $2\times f-2$ y $2\times f-1$ en el arreglo.
Las fichas en el arreglo se especifican en sentido horario partiendo desde la esquina superior izquierda.
Por ejemplo, el cuadrado mostrado en la Figura~\ref{f:cuadrado7} puede ser guardado un arreglo de la siguiente manera.

\begin{center}
% \def\arraystretch{1}
  \setlength\tabcolsep{0.8mm}
\footnotesize
\begin{tabular}{|c|c|c|c|c|c|c|c|c|c|c|c|c|c|c|c|c|c|c|c|c|c|c|c|c|c|c|c|c|c|c|c|c|c|c|c|c|}
  \multicolumn{1}{c}{\tiny \ 0} & 
  \multicolumn{1}{c}{\tiny \ 1} & 
  \multicolumn{1}{c}{\tiny \ 2} & 
  \multicolumn{1}{c}{\tiny \ 3} & 
  \multicolumn{1}{c}{\tiny \ 4} & 
  \multicolumn{1}{c}{\tiny \ 5} & 
  \multicolumn{1}{c}{\tiny \ 6} & 
  \multicolumn{1}{c}{\tiny \ 7} & 
  \multicolumn{1}{c}{\tiny \ 8} & 
  \multicolumn{1}{c}{\tiny \ 9} & 
  \multicolumn{1}{c}{\tiny 10} & 
  \multicolumn{1}{c}{\tiny 11} & 
  \multicolumn{1}{c}{\tiny 12} & 
  \multicolumn{1}{c}{\tiny 13} & 
  \multicolumn{1}{c}{\tiny 14} & 
  \multicolumn{1}{c}{\tiny 15} & 
  \multicolumn{1}{c}{\tiny 16} & 
  \multicolumn{1}{c}{\tiny 17} & 
  \multicolumn{1}{c}{\tiny \ldots} & 
  \multicolumn{1}{c}{\tiny 38} & 
  \multicolumn{1}{c}{\tiny 39} & 
  \multicolumn{1}{c}{\tiny 40} & 
  \multicolumn{1}{c}{\tiny 41} & 
  \multicolumn{1}{c}{\tiny 42} & 
  \multicolumn{1}{c}{\tiny 43} & 
  \multicolumn{1}{c}{\tiny 44} & 
  \multicolumn{1}{c}{\tiny 45} & 
  \multicolumn{1}{c}{\tiny 46} & 
  \multicolumn{1}{c}{\tiny 47} & 
  \multicolumn{1}{c}{\tiny 48} & 
  \multicolumn{1}{c}{\tiny 49} & 
  \multicolumn{1}{c}{\tiny 50} & 
  \multicolumn{1}{c}{\tiny 51} & 
  \multicolumn{1}{c}{\tiny 52} & 
  \multicolumn{1}{c}{\tiny 53} & 
  \multicolumn{1}{c}{\tiny 54} & 
  \multicolumn{1}{c}{\tiny 55}
\\
  \hline
 6& 2& 0& 4& 5& 1& 4& 6& 3& 1& 1& 0& 4& 5& 3& 4& 0& 0& \ldots& 1& 4& 5& 5& 2& 3& 3& 0& 0& 6& 0& 2& 6& 5& 2& 2& 5& 3 \\
  \hline
\end{tabular} 
\end{center}

% \begin{scoreDescription}
%   \score{25} Implementar la función cuadrado, para $n \le 1000$.
%   \score{20} Implementar la función validar, para $n \le 1000$.
%   \score{10} Implementar la función construir, con $n \le 7$.
%   \score{20} Implementar la función construir, para $n \le 16$.
%   \score{25} Implementar la función construir, para $n \le 1000$.
% \end{scoreDescription}

\end{document}
