\documentclass[12pt]{oci}
\usepackage{graphicx}
\usepackage{enumitem}
\usepackage{titling}

\phase{Clasificatoria IOI}

\begin{document}

\begin{titlingpage}
  \begin{center}
  \includegraphics[scale=0.6]{logo.eps}

  \vskip 70pt
  \Large{\bf Olimpiadas Chilenas de Inform\'atica\\}
  \vskip 15pt
  \large{Clasificatoria IOI 2016}
  \vskip 10pt
  \normalsize{\it 16 de Abril, \the\year}


  \vskip 95pt

  \end{center}
  %Este conjunto de problemas est\'a siendo utilizado de manera simultanea en las siguientes sedes:

  %\begin{itemize}[leftmargin=*]
  %\item Arica
  %\item Santiago
  %\item Valparaiso
  %\item Curic\'o
  %\item Talca
  %\item Chill\'an
  %\item Concepci\'on
  %\item Temuco
  %\end{itemize}
\end{titlingpage}

\cleardoublepage


%\subsection*{Informaci\'on General}

%Esta p\'agina muestra informaci\'on general que se aplica a todos los problemas.

%\subsection*{Env\'io de una Soluci\'on}


%\begin{enumerate}
%\itemsep 0em
%\item Los participantes deben enviar {\bf un solo archivo} con el c\'odigo fuente de su soluci\'on.
%\item El nombre del archivo debe tener la extensi\'on \verb+.c+, \verb+.cpp+ o \verb+.pas+ dependiendo de si la soluci\'on est\'a escrita en \verb|C|, \verb|C++| o \verb|Pascal| respectivamente.
%\end{enumerate}

%\subsection*{Casos de Prueba y Subtareas}

%\begin{enumerate}
%\itemsep 0em
%\item La soluci\'on enviada por los participantes ser\'a ejecutada varias veces con distintos casos de prueba.
%\item Cada problema define diferentes subtareas que restringen el problema y se asignar\'a puntaje de acuerdo a la cantidad de subtareas que logre solucionar de manera correcta.
%\item Una soluci\'on puede resolver al mismo tiempo m\'as de una subtarea.
%\item La soluci\'on es ejecutada con cada caso de prueba de manera independiente y por tanto puede fallar en algunas subtareas sin influir en la ejecuci\'on de otras.
%\end{enumerate}

%\subsection*{Entrada}
%\begin{enumerate}
%\itemsep 0em
%\item Toda lectura debe ser hecha desde la {\bf entrada est\'andar} usando, por ejemplo, las funciones \verb+scanf+, \verb+std::cin+ o \verb+readln+ dependiendo del lenguaje escogido.
%\item La entrada corresponde a un solo caso de prueba el cual est\'a descrito en varias l\'ineas dependiendo del problema.
%\item {\bf Se debe asumir que la entrada sigue el formato descrito} en el enunciado de cada problema.
%\end{enumerate}

%\subsection*{Salida}
%\begin{enumerate}
%\itemsep 0em
%\item Toda escritura debe ser hecha hacia la {\bf salida est\'andar} usando, por ejemplo, las funciones \verb+printf+, \verb+std::cout+ o \verb+writeln+ dependiendo del lenguaje escogido.
%\item El resultado del caso de prueba debe ser escrito siguiendo el formato descrito para cada problema.
%\item {\bf El formato de salida debe ser seguido de manera estricta} considerando los espacios especificados, las may\'usculas y min\'usculas.
%\item Toda l\'inea, incluyendo la \'ultima, debe terminar con un salto de l\'inea.
%\end{enumerate}


\end{document}
